\documentclass[a4paper,oneside,onecolumn,12pt]{LegrandOrangeBook}
\usepackage[utf8]{inputenc}
\usepackage[T1]{fontenc}
\usepackage[english]{babel}
\usepackage{longtable}
\usepackage{setspace}
\usepackage{parskip}
\usepackage{minted}
\usepackage{svg}
\usepackage{ragged2e}
\usepackage{markdown}
\usepackage[SC5b,ISBN=000-00-000-0000-0]{ean13isbn}
\usepackage[toc]{appendix}

\markdownSetup{pipeTables,tableCaptions}

\newcommand{\chpt}[1]{\chapter*{#1}\addcontentsline{toc}{section}{#1}}

\hypersetup{
    pdftitle={Creation of the Benex Programming Language}, % Title field
    pdfauthor={Benes Dávid},                               % Author field
    pdfsubject={Szakdolgozat},                             % Subject field
    pdfkeywords={Compiler, Language, x86, Assembly},       % Keywords
    pdfcreator={LaTeX},                                    % Content creator field
}

% Define the color used for highlighting throughout the book
\definecolor{ocre}{RGB}{243, 102, 25}
% Default whitespace from the top of the page to the chapter title on chapter pages
\chapterspaceabove{6.5cm}
% Default amount of vertical whitespace from the top margin to the start of the text on chapter pages
\chapterspacebelow{6.75cm}

\onehalfspacing
\frenchspacing

\hypersetup{
    colorlinks=true,
    linkcolor=blue,
    filecolor=magenta,      
    urlcolor=cyan,
}

\urlstyle{same}

\ExecuteBibliographyOptions{sorting=none}
\addbibresource{manuscript.bib}

\begin{document}

% Title page
\thispagestyle{empty}
\begin{minipage}[c][\textheight][c]{\textwidth}
    {
        \centering
        \includegraphics[keepaspectratio,width=3cm]{SelyeBanner.png} \\
        \vskip0.5cm
        {\LARGE UNIVERZITA J. SELYEHO} \\
        \vskip0.5cm
        {\LARGE SELYE JÁNOS EGYETEM} \\
        \vskip0.5cm
        {\large Fakulta ekonómie a informatiky} \\
        \vskip0.5cm
        {\large Gazdaságtudományi és Informatikai Kar} \\
        \vfill
        {\Huge Creation of the Benex Programming Language} \\
        Szakdolgozat \\
        Benes Dávid \\
        \ISBN \\
        \hfill\the\year{}, Komárom\hfill
    }
\end{minipage}
\cleardoublepage

\begingroup
\makeatletter
\let\ps@plain\ps@empty
\begin{minipage}[c][\textheight][c]{\textwidth}
	{
        \centering
	    {\large UNIVERZITA J. SELYEHO\\SELYE JÁNOS EGYETEM} \\
	    \vskip0.5cm
        {\ NÁZOV FAKULTY\\Fakulta ekonómie a informatiky\\Gazdaságtudományi és Informatikai Kar} \\
        \vfill
        {\Large NÁZOV PRÁCE\\Creation of the Benex Programming Language } \\
        \vfill
        \thispagestyle{empty}
        \begin{tabular}{ll}
            Študijný program:    & Aplikovaná informatika \\
            Tanulmányi program:  & Alkalmazott Informatika \\
            Študijný odbor:      & Informatika \\
            Tanulmányi szak:     & Informatika \\
            Školiteľ:            & László Marák, PhD. \\
            Témavezető:          & László Marák, PhD. \\
            Konzultant:          & László Marák, PhD. \\
            Konzulens:           & László Marák, PhD. \\
            Školiace pracovisko: & Katedra informatiky \\
            Tanszék megnevezése: & Informatikai Tanszék \\
        \end{tabular}
        \vfill
        Označenie typu práce - Szakdolgozat \\
        Benes Dávid \\
        \ISBN \\
        \hfill\the\year{}, Komárno\hfill
	}
\end{minipage}
\endgroup

{
    \hspace*{-2cm}
    \includegraphics[keepaspectratio, width=17cm]{./zadanie/zadanie-EN.pdf}
}

\tableofcontents
\pagebreak
\listoffigures
\addcontentsline{toc}{section}{Ábrák jegyzéke}
\pagebreak

% The begining of the contents
% \chapterimage{kep/header.png}

% \chpt{Feladatkiírás}
% A dolgozat célja egy saját programnyelv készítése a hozzá tartozó fordítóprogrammal és mintaprogramokkal. A szerző egy saját programnyelvet tervez (Benex). Ehhez a programnyelvhez készít egy fordító programot, mely tartalmaz egy frontendet (parser) és egy backendet is (compiler). A compiler futtatható fájlokat hoz létre. A szerző demonstrálja az elkészített compilert néhány példaprogram segítségével.

% % \chapterimage{kep/header2.png}
% \chpt{Opis práce}
% Cieľom tejto práce je vytvoriť programovací jazyk s kompilátorom a vzorovými programami. Autor vyvinie vlastný programovací jazyk (Benex). Pre tento programovací jazyk autor vyvíja kompilátor, ktorý obsahuje frontend (parser) a backend (kompilátor). Kompilátor vytvára spustiteľné súbory. Autor demonštruje kompilátor na niekoľkých príkladoch programov.

% \pagebreak

% \chapterimage{kep/header3.png}
\chpt{Abstrakt}\label{sec:abstrakt}
Cieľom tejto práce je vytvoriť programovací jazyk s kompilátorom a vzorovými programami. Autor vyvinie vlastný programovací jazyk (Benex). Pre tento programovací jazyk autor vyvíja kompilátor, ktorý obsahuje frontend (parser) a backend (kompilátor). Kompilátor vytvára spustiteľné súbory. Autor demonštruje kompilátor na niekoľkých príkladoch programov.

\textbf{Kľúčové slová: Kompilátor, Jazyk, x86, Assembly}

\pagebreak

% \chapterimage{kep/header4.png}
\chpt{Absztrakt}\label{sec:absztrakt}
A dolgozat célja egy saját programnyelv készítése a hozzá tartozó fordítóprogrammal és mintaprogramokkal. A szerző egy saját programnyelvet tervez (Benex). Ehhez a programnyelvhez készít egy fordító programot, mely tartalmaz egy frontendet (parser) és egy backendet is (compiler). A compiler futtatható fájlokat hoz létre. A szerző demonstrálja az elkészített compilert néhány példaprogram segítségével.

\textbf{Kulcsszavak: Compiler, nyelv, x86, Assembly}

\pagebreak

% \chapterimage{kep/header.png}
\chpt{Abstract}\label{sec:abstract}
The aim of this thesis is to create a programming language with compiler and sample programs. The author develops an own programming language (Benex). For this programming language, the author develops a compiler, which includes a frontend (parser) and a backend (compiler). The compiler produces executable files. The author demonstrates the compiler with some example programs.

\textbf{Keywords: Compiler, Language, x86, Assembly}

\chapter*{Motivation and Influences}
\addcontentsline{toc}{chapter}{Motivation and Influences}
\markboth{}{\sffamily\normalsize{Motivation and Influences}}
\markdownInput{motivation.md}

% Begining of the main content
\chapter{Language Theory}
\markdownInput{notes.md}
\markdownInput{x86-0.md}
% The bytes of an instruction
\begin{center}
    \begin{tabular}{|c|c|c|c|c|c|}
        Rex prefix & Opcode & ModR/M & SIB & Displacement & Immediate \\
        1 byte & 1-3 bytes & 1 byte & 1 byte & 1-4 bytes & 1-8 bytes \\
        optional &  & optional & optional & optional & optional
    \end{tabular}
\end{center}
\markdownInput{x86-1.md}
% Truth table of the AND, OR, XOR opeartions
\begin{center}
    \begin{tabular}{|c|c|c|c|c|}
        A & B & A \& B & A $|$ B & A \^ B \\
        0 & 0 & 0 & 0 & 0 \\
        0 & 1 & 0 & 1 & 1 \\
        1 & 0 & 0 & 1 & 1 \\
        1 & 1 & 1 & 1 & 0
    \end{tabular}
\end{center}
\markdownInput{x86-2.md}
% EFLAGS register flags
\begin{center}
    \begin{tabular}{|l|l|l|}
        Shorthand Name & Flag name & Flag description \\
        OF & Overflow Flag & Set if operation overflowed \\
        CF & Carry Flag & Set if operation has the carry bit set \\
        ZF & Zero Flag & Set if operation result was zero \\
        SF & Sign Flag & Set if operation has the negative sign set \\
        PF & Parity Flag & Set if operation has an even number of 1 bits \\
    \end{tabular}
\end{center}
\markdownInput{x86-3.md}
% Conditional jump instructions
\begin{center}
    \begin{tabular}{|l|l|l|}
        Mnemonic & Description & Condition \\
        JA & jump if above & CF == 0 \&\& ZF == 0 \\
        JNBE & jump if not below or equal & CF == 0 \&\& ZF == 0 \\
        JAE & jump if above or equal & CF == 0 \\
        JNB & jump if not below & CF == 0 \\
        JNC & jump if not carry & CF == 0 \\
        JB & jump if below & CF == 1 \\
        JC & jump if carry & CF == 1 \\
        JNAE & jump if not above or equal & CF == 1 \\
        JBE & jump if below or equal & CF == 1 \&\& ZF == 1 \\
        JNA & jump if not above & CF == 1 \&\& ZF == 1 \\
        JCXZ & jump if `cx` zero & CX == 0 \\
        JECXZ & jump if `ecx` zero & ECX == 0 \\
        JRCXZ & jump if `rcx` zero & RCX == 0 \\
        JE & jump if equal & ZF == 1 \\
        JZ &  jump if zero & ZF == 1 \\
        JG & jump if greater & ZF == 0 \&\& SF == OF \\
        JNLE & jump if not less or equal & ZF == 0 \&\& SF == OF \\
        JGE & jump if greater or equal & SF == OF \\
        JNL & jump if not less & SF == OF \\
        JL & jump if less & SF != OF \\
        JNGE & jump if not greater of equal & SF != OF \\
        JLE & jump if less or equal & ZF == 1 \&\& SF != OF \\
        JNG & jump if not greater & ZF == 1 \&\& SF != OF \\
        JNE & jump if not equal & ZF == 0 \\
        JNZ & jump if not zero & ZF == 0 \\
        JNO & jump if not overflow & OF == 0 \\
        JNP & jump if no parity & PF == 0 \\
        JPO & jump if parity odd & PF == 0 \\
        JNS & jump if no sign & SF == 0 \\
        JO & jump if overflow & OF == 1 \\
        JP & jump if parity & PF == 1 \\
        JPE & jump if parity even & PF == 1 \\
        JS & jump if sign & SF == 1 \\
    \end{tabular}
\end{center}
\markdownInput{x86-4.md}


% \chapterimage{kep/header2.png}
\chapter{The Compiler}
\markdownInput{structure.md}
\markdownInput{spec.md}

\chapter*{Conclusion}
\addcontentsline{toc}{chapter}{Conclusion}
Some conclusion of the work.
The goal of our work was to explore the architecture of a compiler and implement a basic example of all the steps required to generate an executable for windows. This basic skeleton can be used to keep building out features for the compiler, and implement any arbitrary feature.

\pagebreak

\chapter*{Resumé}
\addcontentsline{toc}{chapter}{Resumé}
\markboth{}{\sffamily\normalsize{Resumé}}
Some final resume of the work.

% TODO: add this once there is something in the bibliography
% \printbibliography[title=References]
% \addcontentsline{toc}{chapter}{References}

\begin{appendices}
    \markdownInput{functions.md}
\end{appendices}


\end{document}
