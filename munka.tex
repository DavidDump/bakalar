\documentclass[a4paper,oneside,onecolumn,12pt]{LegrandOrangeBook}
\usepackage[utf8]{inputenc}
\usepackage[T1]{fontenc}
\def\magyarOptions{chapterhead=unchanged}
\usepackage[magyar]{babel}
\usepackage{longtable}
\usepackage{setspace}
\usepackage{parskip}
\usepackage{minted}
\usepackage{svg}
\usepackage{ragged2e}
\usepackage{markdown}
\usepackage[SC5b,ISBN=000-00-000-0000-0]{ean13isbn}

\markdownSetup{pipeTables,tableCaptions}

\newcommand{\chpt}[1]{\chapter*{#1}\addcontentsline{toc}{section}{#1}}

% TODO: finish?
\hypersetup{
    pdftitle={Creation of the Benex Programming Language}, % Title field
    pdfauthor={Benes Dávid},                               % Author field
    pdfsubject={Szakdolgozat},                             % Subject field
    pdfkeywords={key1, key2, key3, key4},                  % Keywords
    pdfcreator={LaTeX},                                    % Content creator field
}

% Define the color used for highlighting throughout the book
\definecolor{ocre}{RGB}{243, 102, 25}
% Default whitespace from the top of the page to the chapter title on chapter pages
\chapterspaceabove{6.5cm}
% Default amount of vertical whitespace from the top margin to the start of the text on chapter pages
\chapterspacebelow{6.75cm}

\onehalfspacing
\frenchspacing

\hypersetup{
    colorlinks=true,
    linkcolor=blue,
    filecolor=magenta,      
    urlcolor=cyan,
}

\urlstyle{same}

\ExecuteBibliographyOptions{sorting=none}
\addbibresource{manuscript.bib}

\begin{document}

% Title page
\thispagestyle{empty}
\begin{minipage}[c][\textheight][c]{\textwidth}
    {
        \centering
        \includegraphics[keepaspectratio,width=3cm]{SelyeBanner.png} \\
        \vskip0.5cm
        {\LARGE UNIVERZITA J. SELYEHO} \\
        \vskip0.5cm
        {\LARGE SELYE JÁNOS EGYETEM} \\
        \vskip0.5cm
        {\large Fakulta ekonómie a informatiky} \\
        \vskip0.5cm
        {\large Gazdaságtudományi és Informatikai Kar} \\
        \vfill
        {\Huge Creation of the Benex Programming Language} \\
        Szakdolgozat \\
        Benes Dávid \\
        \ISBN \\
        \hfill\the\year{}, Komárom\hfill
    }
\end{minipage}
\cleardoublepage

\begingroup
\makeatletter
\let\ps@plain\ps@empty
\begin{minipage}[c][\textheight][c]{\textwidth}
	{
        \centering
	    {\large UNIVERZITA J. SELYEHO\\SELYE JÁNOS EGYETEM} \\
	    \vskip0.5cm
        {\ NÁZOV FAKULTY\\Fakulta ekonómie a informatiky\\Gazdaságtudományi és Informatikai Kar} \\
        \vfill
        {\Large NÁZOV PRÁCE\\Creation of the Benex Programming Language } \\
        \vfill
        \thispagestyle{empty}
        \begin{tabular}{ll}
            Študijný program:    & Aplikovaná informatika \\
            Tanulmányi program:  & Alkalmazott Informatika \\
            Študijný odbor:      & Informatika \\
            Tanulmányi szak:     & Informatika \\
            Školiteľ:            & László Marák, PhD. \\
            Témavezető:          & László Marák, PhD. \\
            Konzultant:          & László Marák, PhD. \\
            Konzulens:           & László Marák, PhD. \\
            Školiace pracovisko: & Katedra informatiky \\
            Tanszék megnevezése: & Informatikai Tanszék \\
        \end{tabular}
        \vfill
        Označenie typu práce - Szakdolgozat \\
        Benes Dávid \\
        \ISBN \\
        \hfill\the\year{}, Komárno\hfill
	}
\end{minipage}
\endgroup

{
    \hspace*{-2cm}
    \includegraphics[keepaspectratio, width=17cm]{./zadanie/zadanie-EN.pdf}
}

\tableofcontents
\pagebreak
\listoffigures
\addcontentsline{toc}{section}{Ábrák jegyzéke}
\pagebreak

% The begining of the contents
% \chapterimage{kep/header.png}

\chpt{Feladatkiírás}
A dolgozat célja egy saját programnyelv készítése a hozzá tartozó fordítóprogrammal és mintaprogramokkal. A szerző egy saját programnyelvet tervez (Benex). Ehhez a programnyelvhez készít egy fordító programot, mely tartalmaz egy frontendet (parser) és egy backendet is (compiler). A compiler futtatható fájlokat hoz létre. A szerző demonstrálja az elkészített compilert néhány példaprogram segítségével.

% \chapterimage{kep/header2.png}
\chpt{Opis práce}
Cieľom tejto práce je vytvoriť programovací jazyk s kompilátorom a vzorovými programami. Autor vyvinie vlastný programovací jazyk (Benex). Pre tento programovací jazyk autor vyvíja kompilátor, ktorý obsahuje frontend (parser) a backend (kompilátor). Kompilátor vytvára spustiteľné súbory. Autor demonštruje kompilátor na niekoľkých príkladoch programov.

\pagebreak

% \chapterimage{kep/header3.png}
\chpt{Abstrakt}\label{sec:abstrakt}
Cieľom tejto práce je vytvoriť programovací jazyk s kompilátorom a vzorovými programami. Autor vyvinie vlastný programovací jazyk (Benex). Pre tento programovací jazyk autor vyvíja kompilátor, ktorý obsahuje frontend (parser) a backend (kompilátor). Kompilátor vytvára spustiteľné súbory. Autor demonštruje kompilátor na niekoľkých príkladoch programov.

\textbf{Kľúčové slová: klúč1, klúč2, klúč3}

\pagebreak

% \chapterimage{kep/header4.png}
\chpt{Absztrakt}\label{sec:absztrakt}
A dolgozat célja egy saját programnyelv készítése a hozzá tartozó fordítóprogrammal és mintaprogramokkal. A szerző egy saját programnyelvet tervez (Benex). Ehhez a programnyelvhez készít egy fordító programot, mely tartalmaz egy frontendet (parser) és egy backendet is (compiler). A compiler futtatható fájlokat hoz létre. A szerző demonstrálja az elkészített compilert néhány példaprogram segítségével.

\textbf{Kulcsszavak: kulcs1, kulcs2, kulcs3}

\pagebreak

% \chapterimage{kep/header.png}
\chpt{Abstract}
The aim of this thesis is to create a programming language with compiler and sample programs. The author develops an own programming language (Benex). For this programming language, the author develops a compiler, which includes a frontend (parser) and a backend (compiler). The compiler produces executable files. The author demonstrates the compiler with some example programs.

\textbf{Keywords: key1, key2, key3}

% TODO: intro here
\chapter*{Instroduction}
\addcontentsline{toc}{chapter}{Instroduction}
\markboth{}{\sffamily\normalsize{Instroduction}}
\begin{eBox}
    This is a comment that i might want to add ?? i guess
\end{eBox}
This is the place where i write about the intro...

% TODO: tables are automatically moved to the begining or end of the page, sometimes this makes no sense with how the text flows, insert manual tables

% Begining of the main content
\chapter{Language Theory}
\markdownInput{notes.md}
\markdownInput{x86.md}

% \chapterimage{kep/header2.png}
\chapter{The Compiler}
\markdownInput{structure.md}
\markdownInput{motivation.md}
\markdownInput{spec.md}
\markdownInput{functions.md}

% \chapterimage{kep/header.png}
\chapter*{Conclusion}
\addcontentsline{toc}{chapter}{Conclusion}
Some conclusion of the work.
The goal of our work was to explore the architecture of a compiler and implement a basic example of all the steps required to generate an executable for windows. This basic skeleton can be used to keep building out features for the compiler, and implement any arbitrary feature.

\pagebreak

% \chapterimage{kep/header3.png}
\chapter*{Resumé}
\addcontentsline{toc}{chapter}{Resumé}
\markboth{}{\sffamily\normalsize{Resumé}}
Some final resume of the work.

% TODO: add this once there is something in the bibliography
% \printbibliography[title=References]
% \addcontentsline{toc}{chapter}{References}

\pagebreak
\thispagestyle{empty}

\mbox{}
\vfill

% \begin{Center}
%     \mbox{\vskip1cm}
%     \EANisbn
% \end{Center}

\end{document}
